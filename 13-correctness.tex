\documentclass[aspectratio=169]{beamer}
\usepackage[utf8]{inputenc}
\usepackage[english,russian]{babel}
\usepackage{cancel}
\usepackage{amssymb}
\usepackage{stmaryrd}
\usepackage{cmll}
\usepackage{graphicx}
\usepackage{amsthm}
\usepackage{tikz}
\usepackage{multicol}
\usetikzlibrary{patterns}
\usepackage{chronosys}
\usepackage{proof}
\usepackage{multirow}
\setbeamertemplate{navigation symbols}{}
%\usetheme{Warsaw}

\newtheorem{thm}{Теорема}[section]
\newtheorem{dfn}{Определение}[section]
\newtheorem{lmm}{Лемма}[section]
\newtheorem{exm}{Пример}[section]
\newtheorem{snote}{Пояснение}[section]

\newcommand{\divisible}%                                                     
{\mathrel{\lower.2ex%
\vbox{\baselineskip=0.7ex\lineskiplimit=0pt%
\kern6pt \hbox{.}\hbox{.}\hbox{.}}%
}}

\begin{document}

\newcommand\doubleplus{+\kern-1.3ex+\kern0.8ex}
\newcommand\mdoubleplus{\ensuremath{\mathbin{+\mkern-10mu+}}}

\begin{frame}{}
\LARGE\begin{center}Теорема о корректности формальной арифметики\end{center}
\end{frame}

\begin{frame}{Два вида индукции}
\begin{dfn}\emph{(Принцип математической индукции)}
Какое бы ни было $\varphi(x)$, если $\varphi(0)$ и при всех $x$ выполнено $\varphi(x)\rightarrow \varphi(x')$, то
при всех $x$ выполнено и само $\varphi(x)$.
\end{dfn}

\begin{dfn}\emph{(Принцип полной математической индукции)}
Какое бы ни было $\psi(x)$, если $\psi(0)$ и при всех $x$ выполнено $(\forall t.x < t \rightarrow \psi(x))\rightarrow \psi(x')$, то
при всех $x$ выполнено и само $\psi(x)$.
\end{dfn}

\begin{thm}Принципы математической индукции эквивалентны\end{thm}
\begin{proof}
$(\Rightarrow)$ взяв $\varphi := \psi$, имеем выполненость $\varphi(x)\rightarrow\varphi(x')$, значит, $\forall x.\psi(x)$. \pause\\
$(\Leftarrow)$ возьмём $\psi(x) := \forall t.t\le x\rightarrow\varphi(t)$
\end{proof}
\end{frame}

\begin{frame}{Трансфинитная индукция}
\begin{thm}Принцип трансфинитной индукции. Если для $\varphi(x)$ --- некоторого утверждения
теории множеств --- выполнено:
\begin{enumerate}
\item $\varphi(\varnothing)$
\item Если $\forall u.u \in \upsilon \rightarrow \varphi(u)$, то $\varphi(\upsilon)$ (где $\upsilon$ - это ординал)
\end{enumerate}
то $\forall u.\varphi(u)$
\end{thm}
\end{frame}

\begin{frame}{Индукция для натуральных чисел}
\begin{lmm}Свойство индукции выполнено для натуральных чисел:
если $\varphi(0)$ и $\forall x\in\mathbb{N}_0.f(x) \rightarrow f(x')$, то $\forall x\in\mathbb{N}_0.f(x)$.
\end{lmm}

\begin{proof}
Пусть $\varphi(\varnothing)$ и $\forall u.(u \in \omega) \rightarrow \varphi(u) \rightarrow \varphi(u')$.
Рассмотрим $\tau(n) = \forall u.u \in n \rightarrow \varphi(u)$. Очевидно,
что если $m \in n$, то $\tau(n) \rightarrow \tau(m)$. Значит, выполнены условия принципа
трансфинитной индукции для $\omega$, отсюда $\tau(\omega)$, отсюда $\forall u.(u \in \omega) \rightarrow \varphi(u)$.
\end{proof}
\end{frame}

\begin{frame}{Исчисление $S_\infty$}
\begin{enumerate}                             
\item Язык: связки $\neg$, $\vee$, $\forall$; нелогические символы: $(+)$,$(\cdot)$,$(')$,$0$,$(=)$.
\item Аксиомы: все истинные формулы вида $\theta_1=\theta_2$; все истинные отрицания формул вида $\neg\theta_1=\theta_2$
($\theta_i$ --- термы без переменных).
\item Структурные (слабые) правила:
$$\infer{\zeta\vee\beta\vee\alpha\vee\delta}{\zeta\vee\alpha\vee\beta\vee\delta} \quad\quad
\infer{\alpha\vee\delta}{\alpha\vee\alpha\vee\delta}$$

сильные правила
$$\infer{\alpha\vee\delta}{\delta}\quad
\infer{\neg(\alpha\vee\beta)\vee\delta}{\neg\alpha\vee\delta\quad\neg\beta\vee\delta}\quad
\infer{\neg\neg\alpha\vee\delta}{\alpha\vee\delta}\quad
\infer{(\neg\forall x.\alpha)\vee\delta}{\neg\alpha[x := \theta]\vee\delta}\quad$$

и ещё два правила ...
\end{enumerate}
\end{frame}

\begin{frame}{Ещё правила $S_\infty$}
бесконечная индукция
$$\infer{(\forall x.\alpha)\vee\delta}{\alpha[x:=\overline{0}]\vee\delta
                                  \quad\alpha[x:=\overline{1}]\vee\delta
                                  \quad\alpha[x:=\overline{2}]\vee\delta\quad\dots}$$

сечение
$$\infer{\zeta\vee\delta}{\zeta\vee\alpha\quad\quad\neg\alpha\vee\delta}$$
Здесь: \\
$\alpha$ --- секущая формула \\
Число связок в $\neg\alpha$ --- степень сечения.
\end{frame}

\begin{frame}{Дерево доказательства}
\begin{enumerate}
\item Доказательства образуют деревья.
\item Каждой формуле в дереве сопоставим порядковое число (ординал).
\item Порядковое число заключения любого неструктурного правила строго больше порядкового числа его посылок
(больше или равно в случае структурного правила).

$$\infer{(\forall a.a = a)_\omega}{
   \infer{0 = 0}{}\quad
   \infer{0'= 0'}{\infer{\dots}{\infer{0= 0}{}}}\quad
   \infer{0''= 0''}{\infer{\dots}{\infer{0'= 0'}{\infer{\dots}{\infer{0= 0}{}}}}}\quad\dots
}\quad\quad\infer{(\forall a.a = a)_1}{
   \infer{0 = 0}{}\quad
   \infer{0'= 0'}{}\quad
   \infer{0''= 0''}{}\quad\dots
}$$

%$$\infer{(\neg\neg\forall x.\neg x'=0)_{\omega+1}}{\infer{(\forall x.\neg x' = 0)_\omega}{(\neg 1=0)_1\quad (\neg 2=0)_2 \quad (\neg 3=0)_4 \quad (\neg 4 = 0)_8 \dots}}$$
                 
\item Существует конечная максимальная степень сечения в дереве (назовём её степенью вывода).
\end{enumerate}
\end{frame}

\begin{frame}{Любая теорема Ф.А. --- теорема $S_\infty$}
\begin{thm}Если $\vdash_\text{фа}\alpha$, то $\vdash_\infty|\alpha|_\infty$ \end{thm}
\begin{thm}Если Ф.А. противоречива, то противоречива и $S_\infty$\end{thm}
\begin{exm}Обратное неверно: $$\infer{\forall x.\neg\omega(x,\overline{\ulcorner\sigma\urcorner})}
{\neg\omega(\overline{0},\overline{\ulcorner\sigma\urcorner})\quad\quad
 \neg\omega(\overline{1},\overline{\ulcorner\sigma\urcorner})\quad\quad
 \neg\omega(\overline{2},\overline{\ulcorner\sigma\urcorner})\quad\quad\dots}$$
\end{exm}
\end{frame}

\begin{frame}{Обратимость правил}
\begin{thm}
Если формула $\alpha$ доказана и имеет вид, похожий на заключение правил де Моргана, 
отрицания и бесконечной индукции --- то посылки соответствующих правил могут быть получены из самой 
формулы $\alpha$ доказательством, причём доказательством с не большей степенью и не большим порядком.
\end{thm}
\begin{proof}
Например, формула вида $\neg\neg \alpha\vee\delta$. \pause

Проследим историю $\alpha$; она получена:
\begin{enumerate}
\item ослаблением --- заменим $\neg\neg\alpha$ на $\alpha$ в этом узле и последующих.
\item отрицанием --- выбросим правило, заменим $\neg\neg\alpha$ на $\alpha$ в последующих
\end{enumerate}
Изменённый вывод --- доказательство требуемого.
\end{proof}
\end{frame}

\begin{frame}{Устранение сечений}
\begin{thm}Если $\alpha$ имеет вывод степени $m>0$ порядка $t$, то
можно найти вывод степени строго меньшей $m$ с порядком $2^t$.
\end{thm}

\begin{proof}Трансфинитная индукция по порядку $t$.\begin{enumerate}
\item База. Если $t=0$, то неструктурных правил нет, отсюда $m = 0$.
\item Переход. Рассмотрим заключительное правило.
\begin{enumerate}
\item Не сечение.
\item Сечение, секущая формула --- элементарная.
\item Сечение, секущая формула --- $\neg\alpha$.
\item Сечение, секущая формула --- $\alpha\vee\beta$.
\item Сечение, секущая формула --- $\forall x.\alpha$.
\end{enumerate}
\end{enumerate}
\end{proof}
\end{frame}

\begin{frame}{Случай 1. Не сечение}
$$\infer{\alpha}{\pi_{t_0}\quad\pi_{t_1}\quad\pi_{t_2}\quad\dots}$$
Заменим доказательства посылок $\pi_i$ по индукционному предположению.

\begin{enumerate}
\item Если $m'_i < m_i$, то $\max m'_i < \max m_i$
\item Если $t_i \le t$, то $2^{t_i} \le 2^t$.
\end{enumerate}
\end{frame}

\begin{frame}{Случай 2.4. Сечение с формулой вида $\forall x.\alpha$}
$$\infer{\zeta\vee\delta}{\zeta\vee\forall x.\alpha\quad\quad\neg(\forall x.\alpha)\vee\delta}$$
Причём степень и порядок выводов компонент, соответственно, $(m_1,t_1)$ и $(m_2,t_2)$.
\begin{enumerate}
\item По индукции, вывод $\zeta\vee\forall x.\alpha$ можно упростить до $(m_1',2^{t_1})$.
\item По обратимости, для постоянного $\theta$ можно построить вывод $\zeta\vee\alpha[x := \theta]$ за $(m_1',2^{t_1})$.
\item В формуле $(\neg \forall x. \alpha)\vee\delta$ формула $\neg\forall x.\alpha$ получена
либо ослаблением, либо квантификацией из $\neg\alpha[x := \theta_k]\vee\delta_k$. 
\begin{enumerate}
\item Каждое правило квантификации заменим на:
$$\infer{\zeta\vee\delta_k}{\zeta\vee\alpha[x := \theta_k]\quad\quad(\neg\alpha[x := \theta_k])\vee\delta_k}$$
\item Остальные вхождения $\neg\forall x.\alpha$ заменим на $\zeta$ (в правилах ослабления).
\end{enumerate}
\item В получившемся дереве меньше степень --- так как в $\neg\alpha[x := \theta]$ меньше связок, чем в $\neg\forall x.\alpha$.
\item Нумерацию можно также перестроить.
\end{enumerate}
\end{frame}

\begin{frame}{Теорема об устранении сечений}
\begin{dfn}Итерационная экспонента
$$(a\uparrow)^m(t) = 
  \left\{
    \begin{array}{ll}     t,&m=0\\
                          a^{(a\uparrow)^{m-1}(t)},&m > 0
    \end{array}
  \right.
$$
\end{dfn}
\begin{thm}Если $\vdash_\infty\sigma$ степени $m$ порядка $t$, то найдётся доказательство без сечений
порядка $(2\uparrow)^m(t)$
\end{thm}
\begin{proof}
В силу конечности $m$ воспользуемся индукцией по $m$ и теоремой об уменьшении степени.
\end{proof}
\end{frame}

\begin{frame}{Непротиворечивость формальной арифметики}
\begin{thm}Система $S_\infty$ непротиворечива\end{thm}
\begin{proof} \pause
Рассмотрим формулу $\neg 0=0$. 
Если эта формула выводима в $S_\infty$, то она выводима и в $S_\infty$ без сечений.
Тогда какое заключительное правило? \pause
\begin{enumerate}
\item Правило Де-Моргана? \pause Нет отрицаний дизъюнкции ($\neg(\alpha\vee\beta)\vee\delta$). \pause
\item Отрицание? \pause Нет двойного отрицания ($\neg\neg\alpha\vee\delta$). \pause
\item Бесконечная индукция или квантификация? \pause Нет квантора. \pause
\item Ослабление? \pause Нет дизъюнкции ($\alpha \vee \delta$). \pause
\end{enumerate}

То есть, неизбежно, $\neg 0=0$ --- аксиома, что также неверно.
\end{proof}
\end{frame}

\end{document}
