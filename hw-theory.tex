\documentclass[10pt,a4paper,oneside]{article}
\usepackage[utf8]{inputenc}
\usepackage[english,russian]{babel}
\usepackage{amsmath}
\usepackage{amsthm}
\usepackage{amssymb}
\usepackage{enumerate}
\usepackage{stmaryrd}
\usepackage{cmll}
\usepackage{mathrsfs}
\usepackage[left=2cm,right=2cm,top=2cm,bottom=2cm,bindingoffset=0cm]{geometry}
\usepackage{proof}
\usepackage{tikz}
\usepackage{multicol}
\usepackage{mathabx}

\makeatletter
\newcommand{\dotminus}{\mathbin{\text{\@dotminus}}}

\newcommand{\@dotminus}{%
  \ooalign{\hidewidth\raise1ex\hbox{.}\hidewidth\cr$\m@th-$\cr}%
}
\makeatother

\usetikzlibrary{arrows,backgrounds,patterns,matrix,shapes,fit,calc,shadows,plotmarks}

\newtheorem{definition}{Определение}
\begin{document}

\begin{center}{\Large\textsc{\textbf{Теоретические домашние задания}}}\\
             \it Математическая логика, ИТМО, М3235-М3239, весна 2022 года\end{center}

\section*{Задание №1. Знакомство с классическим исчислением высказываний.}

\begin{enumerate}

\item Будем говорить, что высказывание $\alpha$ выводится из гипотез $\gamma_1, \gamma_2, \dots, \gamma_n$
(и записывать это как $\gamma_1, \gamma_2, \dots, \gamma_n \vdash \alpha$), если существует
такой вывод $\delta_1, \delta_2, \dots, \delta_n$, что $\alpha\equiv\delta_n$, и каждый из $\delta_i$
есть либо гипотеза, либо аксиома, либо получается из каких-то предыдущих высказываний по правилу
Modus Ponens. Несколько гипотез мы можем обозначить какой-нибудь большой буквой середины греческого 
алфавита ($\Gamma,\Delta,\Pi,\Sigma,\Xi$): например, $\Gamma,\alpha,\beta\vdash\sigma$; здесь
$\Gamma$ обозначает какое-то множество гипотез.

Докажите:
\begin{enumerate}
\item $\vdash (A \rightarrow A \rightarrow B) \rightarrow (A \rightarrow B)$
\item $\vdash A \with B \rightarrow B \with A$
\item $\vdash A \with B \rightarrow A \vee B$
\item $\vdash A \rightarrow \neg \neg A$
\item $A \with \neg A \vdash B$
\item $\vdash \neg (A \with \neg A)$
\end{enumerate}

\item Известна теорема о дедукции: $\Gamma, \alpha \vdash \beta$ тогда и только тогда, 
когда $\Gamma \vdash \alpha \rightarrow \beta$. Теорема доказывается конструктивно, то есть
один вывод можно перестроить в другой вывод.
В рамках данного задания разрешается результат её применения вписать как часть другого вывода 
как <<чёрный ящик>> (как макроподстановку). Докажите с её использованием:
\begin{enumerate}
\item $\neg A, B \vdash \neg(A\& B)$
\item $A,\neg B \vdash \neg( A\& B)$
\item $\neg A,\neg B \vdash \neg( A\& B)$
\item $\neg A,\neg B \vdash \neg( A\vee B)$
\item $ A,\neg B \vdash \neg( A\rightarrow B)$
\item $\neg A, B \vdash  A\rightarrow B$
\item $\neg A,\neg B \vdash  A\rightarrow B$
\item $\vdash A \with (B \with B) \rightarrow A \with B$
\item $\vdash (A \rightarrow B) \rightarrow (B \rightarrow C) \rightarrow (A \rightarrow C)$
\item $\vdash (A \rightarrow B) \rightarrow (\neg B \rightarrow \neg A)$ \emph{(закон контрапозиции)}
\item $\vdash A \with B \rightarrow \neg (\neg A \vee \neg B)$ \emph{(правило де Моргана)}
\item $\vdash \neg (\neg A \with \neg B) \rightarrow A \vee B$ \emph{(правило де Моргана)}
\item $\vdash A \with (B \vee C) \rightarrow (A \with B) \vee (A \with C)$ \emph{(дистрибутивность 1)}
\item $\vdash A \vee (B \with C) \rightarrow (A \vee B) \with (A \vee C)$ \emph{(дистрибутивность 2)}
\end{enumerate}

\item Существует несколько аналогов схемы аксиом 10 (аксиомы снятия двойного отрицания). Докажите при любых
высказываниях $\alpha$ и $\beta$:
\begin{enumerate}
\item $\vdash \alpha \vee \neg \alpha$ \emph{(правило исключённого третьего)}
\item $\vdash ((\alpha \rightarrow \beta) \rightarrow \alpha)\rightarrow \alpha$ \emph{(закон Пирса)}
\item без использования 10 схемы аксиом: $((\alpha \rightarrow \beta) \rightarrow \alpha)\rightarrow \alpha \vdash \neg\neg \alpha\rightarrow \alpha$
(то есть, схему аксиом 10 можно заменить на закон Пирса);
\item без использования 10 схемы аксиом: $\alpha \vee \neg \alpha \vdash \neg\neg \alpha\rightarrow \alpha$
(то есть, схему аксиом 10 можно заменить на правило исключённого третьего);
\end{enumerate}

\item Докажите следующие <<странные>> формулы:
\begin{enumerate}
\item $\vdash (A \rightarrow B) \vee (B \rightarrow A)$. В самом деле, получается, что из любых двух 
наугад взятых фактов либо первый следует из второго, либо второй из первого. Например <<если сегодня
пасмурно, то курс матлогики все сдадут на A --- или наоборот, если все сдадут курс матлогики на A,
то сегодня пасмурно>>.
\item Обобщение предыдущего пункта: при любом $n\ge 1$ и любых $\alpha_1, \dots, \alpha_n$ выполнено 
$\vdash (\alpha_1 \rightarrow \alpha_2) \vee (\alpha_2 \rightarrow \alpha_3) \vee \dots \vee (\alpha_{n-1}\rightarrow\alpha_n)
\vee (\alpha_n\rightarrow\alpha_1)$
\item Из противоречия следует всё, что угодно: $\alpha\with\neg\alpha \vdash \beta$
\end{enumerate}

\item В рамках данного задания неравными высказываниями будем называть высказывания
$\alpha$ и $\beta$, имеющие разное количество связок.

Даны высказывания $\alpha$ и $\beta$, причём $\vdash \alpha\rightarrow\beta$ и $\alpha\ne\beta$. 
Укажите способ построения высказывания $\gamma$, такого, что
$\vdash\alpha\rightarrow\gamma$ и $\vdash\gamma\rightarrow\beta$, причём $\alpha\ne\gamma$ и
$\beta\ne\gamma$ .

\item Покажите, что если $\alpha \vdash \beta$ и $\neg\alpha\vdash\beta$, то $\vdash\beta$.
\end{enumerate}

\end{document}
