\documentclass[10pt,a4paper,oneside]{article}
\usepackage[utf8]{inputenc}
\usepackage[english,russian]{babel}
\usepackage{amsmath}
\usepackage{amsthm}
\usepackage{amssymb}
\usepackage{enumerate}
\usepackage{stmaryrd}
\usepackage{cmll}
\usepackage{mathrsfs}
\usepackage[left=2cm,right=2cm,top=2cm,bottom=2cm,bindingoffset=0cm]{geometry}
\usepackage{proof}
\usepackage{tikz}
\usepackage{multicol}
\usepackage{mathabx}

\makeatletter
\newcommand{\dotminus}{\mathbin{\text{\@dotminus}}}

\newcommand{\@dotminus}{%
  \ooalign{\hidewidth\raise1ex\hbox{.}\hidewidth\cr$\m@th-$\cr}%
}
\makeatother

\usetikzlibrary{arrows,backgrounds,patterns,matrix,shapes,fit,calc,shadows,plotmarks}

\newtheorem{definition}{Определение}
\begin{document}

\begin{center}{\Large\textsc{\textbf{Теоретические домашние задания}}}\\
             \it Математическая логика, ИТМО, М3235-М3239, весна 2022 года\end{center}

\section*{Задание №1. Знакомство с классическим исчислением высказываний.}

\begin{enumerate}

\item Будем говорить, что высказывание $\alpha$ выводится из гипотез $\gamma_1, \gamma_2, \dots, \gamma_n$
(и записывать это как $\gamma_1, \gamma_2, \dots, \gamma_n \vdash \alpha$), если существует
такой вывод $\delta_1, \delta_2, \dots, \delta_n$, что $\alpha\equiv\delta_n$, и каждый из $\delta_i$
есть либо гипотеза, либо аксиома, либо получается из каких-то предыдущих высказываний по правилу
Modus Ponens. Несколько гипотез мы можем обозначить какой-нибудь большой буквой середины греческого 
алфавита ($\Gamma,\Delta,\Pi,\Sigma,\Xi$): например, $\Gamma,\alpha,\beta\vdash\sigma$; здесь
$\Gamma$ обозначает какое-то множество гипотез.

Докажите:
\begin{enumerate}
\item $\vdash (A \rightarrow A \rightarrow B) \rightarrow (A \rightarrow B)$
\item $\vdash A \with B \rightarrow B \with A$
\item $\vdash A \with B \rightarrow A \vee B$
\item $\vdash A \rightarrow \neg \neg A$
\item $A \with \neg A \vdash B$
\item $\vdash \neg (A \with \neg A)$
\end{enumerate}

\item Известна теорема о дедукции: $\Gamma, \alpha \vdash \beta$ тогда и только тогда, 
когда $\Gamma \vdash \alpha \rightarrow \beta$. Теорема доказывается конструктивно, то есть
она даёт метод для перестроения одного вывода в другой.
В рамках данного задания разрешается результат её применения вписать как часть другого вывода 
как <<чёрный ящик>> (как макроподстановку). Докажите с её использованием:
\begin{enumerate}
\item $\neg A, B \vdash \neg(A\& B)$
\item $A,\neg B \vdash \neg( A\& B)$
\item $\neg A,\neg B \vdash \neg( A\& B)$
\item $\neg A,\neg B \vdash \neg( A\vee B)$
\item $ A,\neg B \vdash \neg( A\rightarrow B)$
\item $\neg A, B \vdash  A\rightarrow B$
\item $\neg A,\neg B \vdash  A\rightarrow B$
\item $\vdash A \with (B \with B) \rightarrow A \with B$
\item $\vdash (A \rightarrow B) \rightarrow (B \rightarrow C) \rightarrow (A \rightarrow C)$
\item $\vdash (A \rightarrow B) \rightarrow (\neg B \rightarrow \neg A)$ \emph{(закон контрапозиции)}
\item $\vdash A \with B \rightarrow \neg (\neg A \vee \neg B)$ \emph{(правило де Моргана)}
\item $\vdash \neg (\neg A \with \neg B) \rightarrow A \vee B$ \emph{(правило де Моргана)}
\item $\vdash A \with (B \vee C) \rightarrow (A \with B) \vee (A \with C)$ \emph{(дистрибутивность 1)}
\item $\vdash A \vee (B \with C) \rightarrow (A \vee B) \with (A \vee C)$ \emph{(дистрибутивность 2)}
\end{enumerate}

\item Существует несколько аналогов схемы аксиом 10 (аксиомы снятия двойного отрицания). Докажите при любых
высказываниях $\alpha$ и $\beta$:
\begin{enumerate}
\item $\vdash \alpha \vee \neg \alpha$ \emph{(правило исключённого третьего)}
\item $\vdash ((\alpha \rightarrow \beta) \rightarrow \alpha)\rightarrow \alpha$ \emph{(закон Пирса)}
\item Предположим, 10 схема аксиом заменена на две другие схемы аксиом: 
$((\alpha\rightarrow\beta)\rightarrow\alpha)\rightarrow\alpha$ и $\alpha\rightarrow\neg\alpha\rightarrow\beta$.
В этих условиях покажите $\neg\neg \alpha\rightarrow \alpha$.
\item Предположим, 10 схема аксиом заменена на две другие схемы аксиом: 
$\alpha\vee\neg\alpha$ и $\alpha\rightarrow\neg\alpha\rightarrow\beta$.
В этих условиях покажите 
$\neg\neg \alpha\rightarrow \alpha$.
\end{enumerate}

\item Докажите следующие <<странные>> формулы:
\begin{enumerate}
\item $\vdash (A \rightarrow B) \vee (B \rightarrow A)$. В самом деле, получается, что из любых двух 
наугад взятых фактов либо первый следует из второго, либо второй из первого. Например <<выполнено как
минимум одно из утверждений: (а) если сегодня пасмурно, то курс матлогики все сдадут на A; (б) наоборот, если все сдадут курс матлогики на A,
то сегодня пасмурно>>.
\item Обобщение предыдущего пункта: при любом $n\ge 1$ и любых $\alpha_1, \dots, \alpha_n$ выполнено 
$\vdash (\alpha_1 \rightarrow \alpha_2) \vee (\alpha_2 \rightarrow \alpha_3) \vee \dots \vee (\alpha_{n-1}\rightarrow\alpha_n)
\vee (\alpha_n\rightarrow\alpha_1)$
\end{enumerate}

\item В рамках данного задания неравными высказываниями будем называть высказывания
$\alpha$ и $\beta$, у которых нет такого переименования переменных, чтобы их
таблицы истинности совпали. Например, $A$ и $B\with B$ --- равные высказывания,
ведь высказывания $E$ и $E\with E$ имеют одну и ту же таблицу истинности:

\begin{center}\begin{tabular}{c|c}
$E$ & $E \with E$\\\hline
И & И\\\hline
Л & Л
\end{tabular}\end{center}

Однако, высказывания $A$ и $A \rightarrow A$ не равны.

Даны высказывания $\alpha$ и $\beta$, причём $\vdash \alpha\rightarrow\beta$ и $\alpha\ne\beta$. 
Укажите способ построения высказывания $\gamma$, такого, что
$\vdash\alpha\rightarrow\gamma$ и $\vdash\gamma\rightarrow\beta$, причём $\alpha\ne\gamma$ и
$\beta\ne\gamma$ .

\item Покажите, что если $\alpha \vdash \beta$ и $\neg\alpha\vdash\beta$, то $\vdash\beta$.
\end{enumerate}

\section*{Задание №2. Теоремы о корректности и полноте классического исчисления высказываний. Интуиционистская логика.}

\begin{enumerate}

\item Теоремы о корректности и полноте классического исчисления высказываний.
\begin{enumerate}
\item Заполните пробел в доказательстве корректности исчисления высказываний: 
покажите, что если $\vdash \alpha$ и в доказательстве высказывание
$\delta_n$ получено с помощью Modus Ponens из $\delta_j$ и $\delta_k \equiv \delta_j\rightarrow\delta_n$,
то $\models \delta_n$.

\item Покажите, что если $\Gamma \vdash \alpha$, то $\Gamma \models \alpha$.
\item Покажите, что если $\Gamma \models \alpha$, то $\Gamma \vdash \alpha$.
\end{enumerate}

\item Предложите топологические пространства и оценку для пропозициональных переменных,
опровергающие следующие выскзывания:

\begin{enumerate}
\item $A \vee \neg A$ (на лекции приводился пример в $\mathbb{R}$; в данном же задании предложите оценку в 
каком-то другом пространстве, например в $\mathbb{R}^2$)
\item $(((A \rightarrow B) \rightarrow A) \rightarrow A)$
\item $\neg\neg A \rightarrow A$
\item $(A \rightarrow (B \vee \neg B)) \vee (\neg A \rightarrow (B \vee \neg B))$
\item $(A \rightarrow B) \vee (B \rightarrow C) \vee (C \rightarrow A)$
\item $\bigvee_{i=1,n} \left((A_i \rightarrow A_{(i\mod n) + 1}) \with (A_{(i\mod n)+1} \rightarrow A_i)\right)$
\end{enumerate}

\item Доказуемы ли следующие высказывания в интуиционистской логике?
\begin{enumerate}
\item $\neg\neg\neg\neg A \rightarrow \neg\neg A$
\item $\neg A \vee \neg\neg A \vee \neg\neg\neg A$
\item $A \vee B \rightarrow \neg (\neg A \with \neg B)$
\item $\neg (\neg A \vee \neg B) \rightarrow A \with B$
\item $(A \rightarrow B) \rightarrow (\neg A \vee B)$
\item $(\neg A \vee B) \rightarrow (A \rightarrow B)$
\end{enumerate}

\item Известно, что в классической логике любая связка может быть \emph{выражена} как композиция 
конъюнкций и отрицаний: существует схема высказываний, использующая только конъюнкции и отрицания, 
задающая высказывание, логически эквивалентное исходной связке. 
Например, для импликации можно взять $\neg(\alpha\with\neg\beta)$, ведь 
$\alpha\rightarrow\beta\vdash\neg(\alpha\with\neg\beta)$ и $\neg(\alpha\with\neg\beta)\vdash\alpha\rightarrow\beta$. 
Возможно ли в интуиционистской логике выразить через остальные связки:
\begin{enumerate}
\item конъюнкцию?
\item дизъюнкцию?
\item импликацию?
\item отрицание?
\end{enumerate}
Если да, предложите формулу и два вывода. Если нет --- докажите это (например, предложив соответствующую
модель).

\item \emph{Теорема Гливенко.} Обозначим доказуемость высказывания $\alpha$ в классической логике 
как $\vdash_\text{к}\alpha$, а в интуиционистской --- как $\vdash_\text{и}\alpha$. 
Оказывается возможным показать, что какое бы ни было $\alpha$, если $\vdash_\text{к}\alpha$, 
то $\vdash_\text{и}\neg\neg\alpha$. А именно, покажите, что:

\begin{enumerate}
\item Если $\alpha$ --- аксиома, полученная из схем 1--9 исчисления высказываний, то $\vdash_\text{и}\neg\neg\alpha$.
\item $\vdash_\text{и}\neg\neg(\neg\neg\alpha\rightarrow\alpha)$
\item $\neg\neg\alpha,\neg\neg(\alpha\rightarrow\beta) \vdash_\text{и}\neg\neg\beta$
\item Докажите утверждение теоремы ($\vdash_\text{к}\alpha$ влечёт $\vdash_\text{и}\neg\neg\alpha$),
опираясь на предыдущие пункты, и покажите, что классическое исчисление высказываний противоречиво
тогда и только тогда, когда противоречиво интуиционистское.
\end{enumerate}

\item Возможно ли предложить такой набор множеств $S$ из $\mathbb{R}$ (формально: $S \subseteq \mathcal{P}(\mathbb{R})$), 
чтобы при выборе его в качестве истинностного множества $\mathbb{V}$, при сохранении правил вычисления
значений связок для интуиционистской логики, получилась бы полная и корректная модель для классического 
исчисления высказываний?

\item Пусть $S$ --- некоторое множество. Рассмотрим $\mathbb{V} = \mathcal{P}(S)$, 
определим связки так:
$$\begin{array}{ccc}
  \llbracket\alpha \with \beta\rrbracket & = & \llbracket\alpha\rrbracket \cap \llbracket\beta\rrbracket \\
  \llbracket\alpha \vee \beta\rrbracket & = & \llbracket\alpha\rrbracket \cup \llbracket\beta\rrbracket \\
  \llbracket\alpha \rightarrow \beta\rrbracket & = & S\setminus\llbracket\alpha\rrbracket \cup \llbracket\beta\rrbracket \\
  \llbracket\neg\alpha\rrbracket & = & S\setminus\llbracket\alpha\rrbracket  \\
\end{array}$$

Также, будем считать, что $\models \alpha$, если $\llbracket\alpha\rrbracket = S$.

Покажите, что получившееся:
\begin{enumerate}
\item корректная модель классического исчисления высказываний.
Для уменьшения рутинной работы достаточно показать выполнение схемы аксиом 9 и правила Modus Ponens.

\item полная модель классического исчисления высказываний.
\end{enumerate}

\end{enumerate}
\end{document}
